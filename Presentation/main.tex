\documentclass[aspectratio=169]{beamer}

% ---------------- THEME ----------------
\usetheme{Madrid}
\usecolortheme{default}

% ---------------- PACKAGES ----------------
\usepackage[utf8]{inputenc}
\usepackage[T1]{fontenc}
\usepackage[french]{babel}
\usepackage{graphicx}
\usepackage{amsmath, amssymb}
\usepackage{booktabs}
\usepackage{tikz}
\usetikzlibrary{trees, positioning}
\usepackage[table]{xcolor}
\usepackage{colortbl}


% ---------------- TITLE INFO ----------------
\title[Safety Events \& ML]{
Critical Parameter Identification for Safety Events\\
in Commercial Aviation Using Machine Learning
}

\author{ARRIS Yanis \and TEHAMI Safia}
\institute[UVSQ]{Master 2 AMIS}
\date{Janvier 2026}


% ---------------- DOCUMENT ----------------
\begin{document}

% ---------- TITLE SLIDE ----------
\begin{frame}
  \titlepage

  \vspace{0.05cm}
  \footnotesize
  \centering
  \textbf{Article authors:}\\
  HyunKi Lee, Sasha Madar, Santusht Sairam,\\
  Tejas G. Puranik, Michelle Kirby, Olivia J. Pinon,\\
  Dimitri N. Mavris, Alexia P. Payan\\
  \vspace{0.15cm}
  \textit{Aerospace Systems Design Laboratory,\\
  Georgia Institute of Technology\\
  MDPI — Aerospace, 2020}
\end{frame}


% ---------- TABLE DES MATIÈRES ----------
\begin{frame}{Plan de la présentation}
\tableofcontents
\end{frame}

% ===============================
\section{Contexte et problématique}
% ===============================

%-------------------------------------------------
\begin{frame}{Contexte}
\begin{itemize}
  \item La sécurité aérienne s’appuie aujourd’hui sur l’analyse des \textbf{données issues des vols de routine (FOQA)}.
  \item La généralisation des \textbf{capteurs embarqués} entraîne une \textbf{accumulation massive de données},
  atteignant des volumes de l’ordre du téraoctet à l’échelle d’une flotte.
  \item En parallèle, le trafic aérien connaît une forte croissance :
  \begin{itemize}
    \item environ 38 millions de vols en 2018, avec une projection à 90 millions d’ici 2040.
  \end{itemize}
  \item L’enjeu principal est donc de transformer cette masse de données
  en \textbf{connaissance exploitable} pour améliorer la \textbf{prévention des risques}.
\end{itemize}
\end{frame}


%-------------------------------------------------
\begin{frame}{Problématique de sécurité aérienne}
\begin{itemize}
  \item En sécurité aérienne, l’enjeu n’est pas uniquement d’analyser les accidents,
  mais de comprendre \textbf{comment ils se construisent progressivement}.
  
  \item Cette compréhension repose sur :
  \begin{itemize}
    \item l’analyse de la \textbf{chaîne causale} des événements (\textit{pathogenesis}),
    \item l’identification des \textbf{facteurs sous-jacents} (\textit{etiology}).
  \end{itemize}

  \item La sécurité aérienne peut être décrite comme un \textbf{continuum} :
  \begin{center}
    Précurseurs $\rightarrow$ Événements initiateurs $\rightarrow$ Incidents $\rightarrow$ Accidents
  \end{center}

  \item \textbf{Objectif :} détecter et anticiper le passage d’un vol \textbf{nominal}
  vers une situation \textbf{à risque}, \textbf{avant} la survenue d’un incident.
\end{itemize}
\end{frame}


% ===============================
\section{Objectifs}
% ===============================

%-------------------------------------------------
\begin{frame}{Objectif de l'article}
\begin{itemize}
  \item Développer un cadre méthodologique SAFE (Safety Analysis of Flight Events)
  \item Exploiter les données FOQA pour :
  \begin{itemize}
    \item Identifier vols \textbf{à risque} vs vols \textbf{normaux} à partir de \textbf{précurseurs}
    \item Comprendre la transition vers un \textbf{état dangereux}
    \item Intervenir \textbf{avant} la survenue d’un incident
  \end{itemize}
\end{itemize}
\end{frame}


%-------------------------------------------------
\begin{frame}{Pourquoi les approches classiques atteignent leurs limites ?}
\begin{itemize}
  \item Approche souvent \textbf{réactive} : on analyse après l’événement.
  \item Détection basée sur des \textbf{seuils fixes} définis par des experts \textbf{(SMEs)}.
\end{itemize}

\vspace{0.15cm}
\textbf{Limites principales :}
\begin{itemize}
  \item Les seuils ne capturent pas bien les \textbf{dégradations progressives}.(Vision \textbf{binaire} : normal / événement)
  \item Les interactions entre paramètres (effets combinés) sont peu exploitées.
  \item L’analyse devient difficile à généraliser quand les données sont \textbf{massives} et \textbf{hétérogènes}.
\end{itemize}
\end{frame}

% ===============================
\section{Données FOQA et Machine Learning}
% ===============================

%-------------------------------------------------
\begin{frame}{Nature des données utilisées dans l’article}
\begin{itemize}
  \item Données issues des \textbf{vols de routine}, appelées
  \textbf{FOQA} (\textit{Flight Operational Quality Assurance}).
  \item Enregistrements sous forme d’\textbf{observations successives dans le temps}
  couvrant l’ensemble du vol.
  \item Plusieurs centaines à milliers de paramètres, échantillonnés
  jusqu’à \textbf{16 Hz}.
  \item Paramètres provenant de systèmes variés :
  \begin{itemize}
    \item moteurs, atmosphère, dynamique de vol
    \item navigation, commandes de vol, configuration avion
  \end{itemize}
\end{itemize}
\end{frame}


%-------------------------------------------------
\begin{frame}{Pourquoi du Machine Learning ?}
\begin{itemize}
  \item Objectif : passer d’une analyse \textbf{rétrospective} à une analyse plus \textbf{prospective}.
  \item Le ML permet de :
  \begin{itemize}
    \item Exploiter l’\textbf{observability-in-depth}
    \item Capturer \textbf{patterns complexes et non linéaires}
    \item Identifier \textbf{précurseurs faibles} avant dépassement des seuils
  \end{itemize}
\item Finalité : \textbf{anticiper} les événements de sécurité,
plutôt que les analyser uniquement a posteriori

\end{itemize}
\end{frame}

% ===============================
\section{Méthodologie SAFE}
% ===============================

%-------------------------------------------------
\begin{frame}{SAFE : vue d’ensemble}
La méthodologie SAFE (\textit{Safety Analysis of Flight Events}) est structurée
en \textbf{cinq étapes successives} visant à transformer les données FOQA
en information exploitable pour la sécurité aérienne.
\begin{enumerate}
  \item Prétraitement des données
  \item Réduction de la dimensionnalité
  \item Génération des vecteurs de caractéristiques
  \item Classification
  \item Post-traitement et interprétation
\end{enumerate}
\end{frame}

%-------------------------------------------------
\begin{frame}{Étape 1 : Prétraitement des données}
Objectif : nettoyer les données FOQA brutes afin d’obtenir un jeu exploitable.

\begin{itemize}
  \item \textbf{suppression} : suppression des paramètres trop incomplets
  \item \textbf{Replacement} : remplacement des valeurs manquantes pour conserver les vols
\end{itemize}

\vspace{0.25cm}

\textbf{Exemples (inspirés de l’article) :}

\vspace{0.1cm}

\textbf{suppression :}
\[
\begin{array}{c|ccc}
\text{Temps} & A & B & C \\
\hline
10 & 0 & 0.05 & \text{DNE} \\
11 & 0 & 0.02 & \text{DNE} \\
12 & 1 & 0.01 & \text{DNE}
\end{array}
\;\Rightarrow\;
\begin{array}{c|cc}
\text{Temps} & A & B \\
\hline
10 & 0 & 0.05 \\
11 & 0 & 0.02 \\
12 & 1 & 0.01
\end{array}
\]

\vspace{0.15cm}

\textbf{Replacement :}
\[
\begin{array}{c|cc}
\text{Temps} & A & B \\
\hline
13 & 1 & \text{Empty} \\
14 & 1 & \text{Empty}
\end{array}
\;\Rightarrow\;
\begin{array}{c|cc}
\text{Temps} & A & B \\
\hline
13 & 1 & 0.0000 \\
14 & 1 & 0.0000
\end{array}
\]

\footnotesize
\textit{Les paramètres trop incomplets sont supprimés, les valeurs manquantes ponctuelles sont imputées.}
\end{frame}


%-------------------------------------------------
\begin{frame}{Étape 2 : Réduction de la dimensionnalité}

Les données FOQA présentent de nombreuses redondances dues à
des corrélations physiques entre paramètres.

\vspace{0.2cm}

\begin{columns}[T,onlytextwidth]

% -------- MATRICE DE CORRÉLATION --------
\begin{column}{0.48\textwidth}
\centering
\textbf{Matrice de corrélation}

\[
\begin{array}{c|cccc}
 & A & B & C & D \\
\hline
A & 1 & \cellcolor{green!25}0.996 & \cellcolor{green!25}0.994 & 0.707 \\
B & 0.996 & 1 & \cellcolor{green!25}0.997 & \cellcolor{green!25}0.991 \\
C & 0.994 & 0.997 & 1 & 0.947 \\
D &0.707 & 0.991 & 0.947 & 1 \\
\end{array}
\]

\footnotesize
\textit{Les corrélations $> 0.99$ indiquent des paramètres redondants.}
\end{column}

% -------- FLÈCHE --------
\begin{column}{0.08\textwidth}
\centering
\vspace{1.6cm}
$\Longrightarrow$
\end{column}

% -------- REGROUPEMENT --------
\begin{column}{0.44\textwidth}
\centering
\textbf{Regroupement des paramètres}

\[
\begin{array}{c|c}
\text{Paramètre} & \text{Paramètres corrélés} \\
\hline
A & [B, C] \\
B & [C, D] \\
C & [\,] \\
D & [E] \\
E & [\,] \\
F & [\,]
\end{array}
\]

\footnotesize
\textit{Un seul paramètre est conservé par groupe de forte corrélation
(seuil : $0.99$), sur la base de l’expertise métier.}
\end{column}

\end{columns}

\end{frame}


%-------------------------------------------------
\begin{frame}{Étape 3 : Génération des vecteurs de caractéristiques}
Les observations successives dans le temps sont transformées
pour être compatibles avec la classification.

\begin{itemize}
  \item Chaque vol est représenté par un \textbf{vecteur de caractéristiques}
\end{itemize}

\textbf{Stratégies temporelles proposées :}
\begin{itemize}
  \item sélection d’un instant unique
  \item sélection de timestamps consécutifs avant l’événement
  \item sélection de timestamps échelonnés
\end{itemize}
 
\end{frame}
%-------------------------------------------------
\begin{frame}{Étape 3 : Génération des vecteurs de caractéristiques\\(un seul timestamp)}

\begin{columns}[T,onlytextwidth]

% -------- COLONNE GAUCHE --------
\begin{column}{0.52\textwidth}
\centering
\textbf{Données temporelles par vol}

\vspace{0.2cm}

\textbf{Flight ID : 123123}

\vspace{0.05cm}
\footnotesize
\[
\begin{array}{ccc}
\ C & E & F \\
\hline
\rowcolor{green!20}10 & 0 & 0.0534 \\
11 & 0 & 0.0276 \\
12 & 1 & 0.0129 
\end{array}
\]

\vspace{0.35cm}

\textbf{Flight ID : 123126}

\vspace{0.05cm}
\footnotesize
\[
\begin{array}{ccc}
\ C & E & F \\
\hline
\rowcolor{orange!25}16 & 0 & 0.0276 \\
17 & 1 & 0.0129
\end{array}
\]
\end{column}

% -------- FLÈCHE --------
\begin{column}{0.06\textwidth}
\centering
\vspace{2.2cm}
$\Longrightarrow$
\end{column}

% -------- COLONNE DROITE --------
\begin{column}{0.42\textwidth}
\centering
\textbf{Feature Vector Matrix}

\vspace{0.15cm}
\footnotesize
\[
\begin{array}{c|ccc}
\text{Flight ID} & C & E & F \\
\hline
\rowcolor{green!20}123123 & 10 & 0 & 0.0534 \\
\rowcolor{orange!25}123126 & 16 & 0 & 0.0276 \\
148279 & 12 & 0 & 0.0386 \\
213219 & 8 & 0 & 0.1108 \\
459876 & 21 & 1 & 0.0041
\end{array}
\]

\vspace{0.2cm}
\scriptsize
\textit{Un instant représentatif est sélectionné par vol.}
\end{column}

\end{columns}

\end{frame}

%-------------------------------------------------
\begin{frame}{Étape 3 : Génération des vecteurs de caractéristiques\\(timestamps multiples)}

\begin{columns}[T,onlytextwidth]

% -------- COLONNE GAUCHE --------
\begin{column}{0.52\textwidth}
\centering
\textbf{Données temporelles par vol}

\vspace{0.2cm}

\textbf{Flight ID : 123123}

\vspace{0.05cm}
\footnotesize
\[
\begin{array}{ccc}
\ C & E & F \\
\hline
\rowcolor{green!20}10 & 0 & 0.0534 \\
\rowcolor{green!20}11 & 0 & 0.0276 \\
12 & 1 & 0.0129 
\end{array}
\]

\vspace{0.35cm}

\textbf{Flight ID : 123126}

\vspace{0.05cm}
\footnotesize
\[
\begin{array}{ccc}
\ C & E & F \\
\hline
\rowcolor{orange!25}16 & 0 & 0.0276 \\
\rowcolor{orange!25}17 & 1 & 0.0129
\end{array}
\]
\end{column}

% -------- FLÈCHE --------
\begin{column}{0.06\textwidth}
\centering
\vspace{2.2cm}
$\Longrightarrow$
\end{column}

% -------- COLONNE DROITE --------
\begin{column}{0.42\textwidth}
\centering
\textbf{Feature Vector Matrix}

\vspace{0.15cm}
\scriptsize
\[
\begin{array}{c|cccccc}
\text{Flight ID} & C_1 & E_1 & F_1 & C_2 & E_2 & F_2 \\
\hline
\rowcolor{green!20}123123 & 10 & 0 & 0.0534 & 11 & 0 & 0.0276 \\
\rowcolor{orange!25}123126 & 16 & 0 & 0.0276 & 17 & 1 & 0.0129 \\
148279 & 12 & 0 & 0.0386 & 13 & 1 & 0.0021 \\
213219 & 8 & 0 & 0.1108 & 9 & 1 & 0.0098 \\
459876 & 21 & 1 & 0.0041 & 22 & 1 & 0.0017
\end{array}
\]

\vspace{0.2cm}
\scriptsize
\textit{Plusieurs instants consécutifs sont concaténés pour capturer la dynamique temporelle.}
\end{column}

\end{columns}

\end{frame}


%-------------------------------------------------
\begin{frame}{Étape 4 : Classification et critères de sélection}
La sélection de l’algorithme de classification repose sur plusieurs critères définis dans l’article :
\begin{itemize}
  \item Capacité à détecter correctement les événements de sécurité (accuracy)
  \item Robustesse face aux paramètres corrélés
  \item Capacité à traiter des données de grande dimension
  \item Robustesse face au surapprentissage
  \item Capacité à gérer des paramètres hétérogènes
  \item Simplicité du réglage des hyperparamètres
\end{itemize}

\vspace{0.2cm}
Ces critères sont utilisés pour comparer plusieurs algorithmes de classification.
\end{frame}

\begin{frame}{Comparaison des algorithmes de classification}
\centering
\begin{tabular}{lcccccc}
\toprule
\textbf{Algorithme} & Acc. & Corr. & Dim. & Overf. & Robust. & Tuning \\
\midrule
Boosting Ensemble & X & X & X & X &  &  \\
Decision Tree     & X & X &   &   &  &  \\
K-NN              &   & X & X &   & X &  \\
Bayes Class.       &   &   & X &   & X & X \\
\textbf{Random Forest} & \textbf{X} & \textbf{X} & \textbf{X} & \textbf{X} & \textbf{X} &  \\
\bottomrule
\end{tabular}

\vspace{0.25cm}
\footnotesize
\textit{Acc. : capacité de prédiction \quad
Corr. : résistance aux paramètres corrélés \quad
Dim. : gestion de données de grande dimension \quad
Overf. : résistance au surapprentissage \quad
Robust. : capacité à gérer des paramètres hétérogènes \quad
Tuning : facilité de réglage des hyperparamètres}



\vspace{0.15cm}
\textbf{Conclusion :} Random Forest présente le meilleur compromis global,
en particulier grâce à sa résistance au surapprentissage et à sa capacité
d’interprétation.
\end{frame}


%-------------------------------------------------


\begin{frame}{Random Forest : exemple concret (Tire Speed Event)}

\centering

\begin{tikzpicture}[
  level distance=1.15cm,
  level 1/.style={sibling distance=5.8cm},
  level 2/.style={sibling distance=3.2cm},
  level 3/.style={sibling distance=1.8cm},
  every node/.style={
    draw, rectangle, rounded corners,
    align=center,
    font=\scriptsize,
    inner sep=2.5pt
  },
  edge from parent/.style={draw,-},
  leaf/.style={font=\scriptsize, inner sep=2pt}
]

% --- ARBRE ---
\node (root) {Paramètre principal\\\textbf{Vitesse sol}}
  child {node {Condition 1\\\textbf{Poids élevé ?}}
    child {node {Condition 1.1\\\textbf{Densité faible ?}}
      child {node[leaf] {\textcolor{red}{Event}}}
      child {node[leaf] {No Event}}
    }
    child {node {Condition 1.2\\Poids normal}
      child {node[leaf] {No Event}}
    }
  }
  child {node {Condition 2\\\textbf{Poussée élevée ?}}
    child {node {Condition 2.1\\Fin de décollage}
      child {node[leaf] {\textcolor{red}{Event}}}
      child {node[leaf] {No Event}}
    }
    child {node {Condition 2.2\\Poussée normale}
      child {node[leaf] {No Event}}
      child {node[leaf] {No Event}}
    }
  };

% --- COMMENTAIRE CÔTÉ DROIT ---
\node[
  right=1.2cm of root,
  align=left,
  font=\footnotesize,
  draw=none
] {
\textbf{Lecture de l’arbre :}\\
Branche gauche $\rightarrow$ \textbf{Oui}\\
Branche droite $\rightarrow$ \textbf{Non}
};

\end{tikzpicture}


\vspace{0.3cm}

\begin{block}{Principe de Random Forest}
\footnotesize
\begin{itemize}
  \item Chaque \textbf{arbre de décision} apprend une suite de règles simples
  basées sur des seuils de paramètres.
  \item Les arbres sont construits à partir de \textbf{sous-ensembles différents}
  des données et des paramètres.
  \item Chaque arbre produit une décision \textit{Event / No Event}.
  \item La décision finale est obtenue par \textbf{vote majoritaire}
  de l’ensemble des arbres.
\end{itemize}
\end{block}

\end{frame}


%-------------------------------------------------

\begin{frame}{Étape 5 : Évaluation et interprétation}
L’évaluation du modèle repose sur :
\begin{itemize}
  \item la \textbf{matrice de confusion}
  \item le \textbf{F1-score}, particulièrement adapté aux données déséquilibrées
\end{itemize}

\vspace{0.2cm}

\textbf{Matrice de confusion :}
\[
\begin{array}{c|cc}
 & \text{Event réel} & \text{Non-Event réel} \\
\hline
\text{Event prédit} & TP & FP \\
\text{Non-Event prédit} & FN & TN
\end{array}
\]

\vspace{0.2cm}

\textbf{Interprétation :}
\begin{itemize}
  \item \textbf{TP (True Positive)} : événement correctement détecté par le modèle
  \item \textbf{FP (False Positive)} : fausse alerte (événement prédit mais absent)
  \item \textbf{FN (False Negative)} : événement manqué par le modèle (cas critique en sécurité)
  \item \textbf{TN (True Negative)} : vol sans événement de sécurité, correctement identifié
\end{itemize}

\end{frame}


%-------------------------------------------------
\begin{frame}{F1-score : formules et exemple}
Formules utilisées :
\[
Precision = \frac{TP}{TP + FP}
\quad
Recall = \frac{TP}{TP + FN}
\]
\[
F1 = \frac{2 \cdot Precision \cdot Recall}{Precision + Recall}
\]

\vspace{0.2cm}
\textbf{Exemple (Tire Speed Event, article) :}
\begin{itemize}
  \item $TP=55$, $FP=1$, $FN=0$
\end{itemize}

\[
Precision \approx 0.982 \quad Recall = 1 \quad F1 \approx 0.991
\]
\end{frame}

% ===============================
\section{Étude de cas}
% ===============================

%-------------------------------------------------
\begin{frame}{Étude de cas : Tire Speed Event}

\textbf{Définition :}  
Le \textbf{Tire Speed Event} correspond au dépassement de la vitesse certifiée
des pneus lors du décollage.

\vspace{0.2cm}

\textbf{Résultats principaux :}
\begin{itemize}
  \item Performances très élevées du modèle
  \item \textbf{F1-score proche de 1}
\end{itemize}

\vspace{0.15cm}

\textbf{Paramètres critiques identifiés par SAFE :}
\begin{itemize}
  \item Vitesse sol
  \item Poids de l’avion
  \item Densité de l’air
  \item Poussée moteur (thrust)
\end{itemize}

\vspace{0.15cm}

\textbf{Apport du Machine Learning :}  
La poussée moteur n’est pas toujours mise en avant dans les descriptions
classiques, mais l’analyse ML montre qu’elle devient \textbf{significative}
lorsqu’elle est considérée \textbf{en interaction} avec le poids de l’avion
et la densité de l’air.

\end{frame}


% ===============================
\section{Discussion des limites et perspectives}
% ===============================

\begin{frame}{Discussion \& perspectives}
\centering
\vfill
\Large \textbf{Bilan des performances, limites}\\
\vspace{0.3cm}
\Large \textbf{et pistes d’amélioration}
\vfill
\end{frame}


%-------------------------------------------------
\begin{frame}{Performances et portée du cadre}
\begin{itemize}
    \item Le cadre SAFE obtient de bonnes performances globales sur plusieurs types d’événements de sécurité.
    \item Les scores F1 restent élevés, entre \textbf{0,85 et 0,99}.
    \item Le cadre est implémentable en pratique et transférable à d’autres contextes.
    \item Son efficacité dépend fortement de la qualité des définitions d’événements.
\end{itemize}
\end{frame}
%-------------------------------------------------
\begin{frame}{Limites liées aux définitions et à la causalité}
\begin{itemize}
    \item Certains événements restent difficiles à analyser, notamment le Roll Event et le Landing Distance Event.
    \item Ces difficultés proviennent de définitions imprécises ou non standardisées.
    \item Une tension existe entre définitions fonctionnelles et définitions physiques.
    \item Les algorithmes statistiques peuvent confondre corrélation et causalité, introduisant de l’incertitude dans la chaîne causale.
\end{itemize}
\end{frame}

%-------------------------------------------------
\begin{frame}{ML et expertise métier : une approche hybride}
\begin{itemize}
    \item Le Machine Learning seul ne suffit pas pour capturer la dynamique réelle des événements.
    \item L’analyse algorithmique doit être complétée par l’expertise métier (SME).
    \item Les experts jouent un rôle clé pour contextualiser les résultats et valider leur cohérence physique.
    \item Une interaction continue entre ML et SME est nécessaire
pour garantir la validité opérationnelle des résultats.
\end{itemize}
\end{frame}

%-------------------------------------------------
\begin{frame}{Recommandations et perspectives}
\begin{itemize}
    \item Adapter la conception des événements en réduisant l’usage de seuils fixes et l’agrégation excessive de sous-événements.
    \item Adapter les algorithmes afin de limiter les facteurs corrélés mais non causaux.
    \item Explorer des alternatives aux modèles basés sur des arbres de décision.
    \item Le ML ouvre la voie à une analyse plus prospective des événements, à condition d’une utilisation critique et raisonnée.
\end{itemize}
\end{frame}
%-------------------------------------------------

\begin{frame}
\centering
\vfill
\Huge \textbf{Merci de votre attention}\\
\vspace{0.4cm}
\Large \textit{À vos questions}
\vfill
\end{frame}



\end{document}
